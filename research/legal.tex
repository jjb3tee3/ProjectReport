\subsection{Legal Issues}
As probe requests are broadcast from wireless devices, monitoring these frames is not an illegal activity. In UK law; however, once you start monitoring the data passing on the network it becomes a legal- and ethical- issue. Being privy to data packets addressed to other users may leave the attacker open to prosecution under The Regulation of Investigatory Powers Act \cite{research:uk_law1}, but the person broadcasting the data may also be liable under the Data Protection Act 1998 \cite{research:uk_law2}. The Regulation of Investigatory Powers Act outlines the various public authorities permitted to use different types of investigative techniques, including interception of a communication and use of communication data \cite{research:uk_law3}, both of which monitoring network data traffic would fall under.

It is interesting to note that a recent precedent \cite{research:uk_law4} in Illinois, USA, determined that monitoring traffic passing on an open network was not against the law as their protocol removed the payload from the data packets, arguing that their application acted in the same way that network cards currently do in forwarding- or ignoring- any packets not addressed to them. 

Between 2008 and 2010 Google used Kismet in their Street Car software \cite{research:uk_law5} to capture 200 gigabytes of personal data from unencrypted network traffic whilst they set out to map the world. Google's wardriving campaign was initially to aid their geolocation application's ability to provide location-based services, by mapping the location of wireless access points; however, a ``rogue engineer'' had a design document signed off that detailed how extra information about browsing habits may be of interest. Extra information, it turns out, included passwords, emails, and video and audio data. Initially Google denied this allegation, though May 14th 2014 Google publically acknowledged that the code was ``mistakenly'' added to the software. The rogue engineer was eventually named as Marius Milner, the developer of NetStumbler- a Windows tool for detecting wireless networks. Interestingly in this case, albeit from the US, Google was found not to have broken their Wiretapping Act of 1986.

The US was not the only country to decide that Google's collection of payload data was possibly unlawful, France, Canada and the Netherlands have also undertaken inquiries.

The technique Google used is not difficult, and not too dissimilar to what is trying to be achieved in this project. The final implementation could quite easily be changed to capture packet payload data; however, as noted above, this would be against UK law.