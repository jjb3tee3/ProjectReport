\subsubsection{Legal Issues}
As probe requests are broadcast from wireless devices, monitoring these frames is not an illegal activity. In UK law; however, once you start monitoring the data passing on the network it becomes a legal- and ethical- issue. Being privy to data packets addressed to other users may leave the attacker open to prosecution under a number of acts, namely the Computer Misuse Act 1990 [3] and the Data Protection Act 1998 [4]. The Regulation of Investigatory Powers Act [2] outlines the various public authorities permitted to use different types of investigative techniques, including interception of a communication and use of communication data [5], both of which monitoring network data traffic would fall under.

It is interesting to note that a recent precedence[1] in Illinois, USA, determined that monitoring traffic passing on an open network was not against the law as their protocol removed the payload from the data packets, arguing that their application acted in the same way that network cards currently do in forwarding- or ignoring- any packets not addressed to them. 
