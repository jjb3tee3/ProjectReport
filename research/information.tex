\subsubsection{Innocuous Information}
Often when using the internet we disregard just how personal seemingly meaningless actually is when combined. In the introduction I touched upon browser fingerprinting as a technique for circumventing UK cookie laws and uniquely identifying users, with a surprisingly high amount of accuracy. 

It has been noted that only 33bits of information are needed to uniquely identify a single person on the planet [1]. This value is calculated by taking the total number of people on the planet n, and calculating lg(n) to get the value x where x is the number of yes/no decisions required to uniquely identify someone. Of course, questions with more answers mean more entropy, so where a simple question of gender will give you 1 bit of entropy, knowing a person’s postcode will  give you an amount relative to the number of inhabitants of that particular place in comparison to the world.

There has been research in the past looking at identifying people through two sets of de-identified data by linking common fields, thus identifying the person [2]. The same author went on to set up a (now defunct) website that determined how unique you were based only gender, date of birth and postcode [3] and the results can be calculated easily. Given the first part of my home postcode narrows me down to 15,630 in the world. Take into consideration my gender and we can split that figure in half to 7815. Assuming for simplicities sake that the average life expectancy in the UK is 100 (it’s approx. 80), we are given 36,500 as the number of possible birthdates. This means there is a 21\% chance that someone will share the same birthday at me, or, an 79\% chance that I am uniquely identifiable through only three pieces of information. Of course, this is a very crude approximation and does not take into consideration a number of variables, and simplifies others. Papers that have attempted this with US census data have found it to be closer to 63\% [5].

The information used in the above example are things a majority of the general public would not think twice about entering in to an online form.

Arguably one of the most identifying information structures about a user is their browser history. Attacks have been established, that browser companies have been aware of for over a decade [4], that simply make use of the text colour of a link to determine whether your browser has visited a website before or not. This seems fairly trivial to start with, but when you pair it with a malicious website that checks against the top 1 million websites using JavaScript then it starts to become a viable method of identification. Unfortunately this loophole has been closed by browser developers, much to the scorn of the web design industry. Although, its death did bring about the resurrection of an old attack designed to determine the user's web history- cache timing. 
