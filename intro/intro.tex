\section {INTRODUCTION}
\subsection{Background}
Cyber security has become prevalent in the media over the recent years with various whistleblowers allowing the general public an insight into just how government agencies across the world, Cyber security has become prevalent in the media over the recent years with various whistleblowers allowing the general public an insight into just how government agencies across the world, such as GCHQ (Government Communications Headquarters) and the NSA (National Security Agency), have developed software that allows them to monitor digital communications. 

The software developed takes advantage of zero-day vulnerabilities, and network infrastructure.

Mobile wireless devices, by design, roam in order to discover WiFi networks to transfer data over in an attempt to lower that sent over the cellular network. Devices will probe for previously used wireless access points by periodically emit probe request frames to determine whether the access point is in range.

The 802.11 is an inherently insecure base in which a large majority of enterprise applications are founded upon. Attempts have been made to secure networks by using various types of encryption, for example Wired Equivalent Policy and WiFi Protected Access (WPA), however these standards only encrypt the data portion of the exchange. Management frames are left open and unencrypted which offers opportunity for exploitation in a number of ways, from getting unsuspecting users to connect to a faked access point to collecting data based on access points a mobile device has connected to through the period beacon frames.

This project ties together a number of attacks on the 802.11 standard into one automated system as a way of attacking mobile devices.

\subsection{Objectives}
This aim of this project to expose the inherent security vulnerability that unencrypted and WEP encrypted networks leave behind on mobile wireless devices whilst they pass by in the pocket of the owner. This will be achieved by developing software capable of responding to 802.11 probe request frames emitted by roaming smartphones in an attempt to masquerade as the requested access point. 

Once a connection with a device has been established the laptop running the software will create a bridge between the soft access point and the hardware to allow for traffic to pass normally, and HTTP traffic to be analysed. This will help to determine what information can be gained from various apps running in the background. 

Finally, I want to look into what steps can be taken to prevent this attack happening at root level, and the possible ways of adding security at application level to notify users of hijacking attempts.

\subsection{Content}
There isn't any.
\clearpage