\section{Requirements}
This requirement specification uses keywords to indicate requirement levels as defined in RFC 2119 (appendix X[1]). Consideration has been taken to follow principles outlined in the IEEE Recommended Practice for Software Requirements Specification [2] whilst being mindful that they are included as part of a larger document.

\subsection{Scope}
The requirements put forth are for the Honeypot application that will be used to create spoofed access points based on probe request frames broadcast from passing mobile wireless devices, and the Android application that leaks personal data. The requirements are split into separate sections that define the hardware and software requirements for the honeypot application, then moves on to define the specific requirements for the honeypot application itself, and the leaky Android application.

\subsection{Overall Requirements}
\subsubsection{User Interfaces}
The application will have no graphical user interface (GUI) and will provide relevant, contextual, information to the user through a simple console interface. Relevant information is defined below as part of the functional requirements. Usage of the application will require some prior knowledge of the 802.11 standard and its naming conventions.

Messages displayed to the user must be configurable in terms of debug, information, warning and error. Notification of the desired level should be configurable while starting the application, but not during application execution.

Viewing of HTTP data may be performed by an external application, such as Wireshark. Using such an application will require previous experience of creating filters to find the desired traffic.

\subsubsection{Hardware Interfaces}
Due to the application needing to be run in a high traffic area, to make it as effective as possible, it will need to be executed on a laptop. The Operating System may either installed as the primary, or, run in a Virtual Machine.

The honeypot application requires a wireless adaptor that conforms to the standards determined in section \ref{research:hardware}. To summise, they are as follows: possesses the ability to be put into monitor mode to capture traffic on the network, support for packet injection into the network, and support from the chipset manufacturer toward the open source community, as this will give us greater confidence in the drivers for Linux supporting the capabilities required, and working out-of-the-box.

Considering these points, and having used it during the research portion of this project, the Alfa AWUS036H (pictured in figure \ref{research:fig-alfa} on page \pageref{research:fig-alfa}) has been selected as the wireless adaptor. As such, this must be used when implementing the honeypot application.

A laptop will be required to run the application, and it will need at least one spare USB port for the wireless adaptor to be plugged in to and a spare ethernet port to connect to a wireless router.

The wireless network must have the SSID hidden, must have no encryption and must be connected to the laptop via ethernet cable. This is required so that the spoofed AP may bridge its traffic and the users of the spoofed AP will be able to access the internet

\subsubsection{Software Interfaces}
The application must run on Backtrack 5 Revision 3 (B5R3) which is available from the Backtrack website[1]. The application may run on Kali Linux which is a derivative of Backtrack developed by the same team, but with a Debian foundation instead of Red Hat.

The Operating System must be either installed as the primary, or installed on a Virtual Machine. If the Operating System is installed on a Virtual Machine then the virtualisation software used must be VirtualBox V4.3.0 [2]. The network adapter in VirtualBox must be set to bridged and utilize the ethernet adapter of the host laptop.

Air-ng may be utilized to create the spoof access point once the requirements of the AP have been determined. This software suite comes pre-installed on the required Operating System. 

\subsubsection{Communications Interface}
As this application is designed to exploit vulnerabilities in the 802.11 standard, 802.11 will be required as a protocol. At the higher level HTTP traffic data will be analysed for information.

\subsubsection{Operations}
Once the application has been started by the user it must be entirely autonomous, acting upon user defined parameters at execution. The only user interaction should be exiting the application, which on signal it shall gracefully exit.

\subsubsection{Site Adaptation Requirements}
This application must only be run only in controlled wireless lab conditions due to the inability to get informed consent of the participants when running publicly. Under no circumstances should the application be executed in a public location, without a predefined SSID filter.

The application will require the host machine to be connected to an internet source via ethernet in order for it to successfully bridge traffic.

\newpage
\subsubsection{Functional Requirements}
This section outlines the requirements for each individual application. Section \ref{requirements:specific} delves in to the details of each requirement and displays them in a numbered format to make it easier to test individual requirements at the end.

\subsubsection*{Honeypot Application}
The application must:-
\begin{itemize}
	\item Take user defined parameters:
		\begin{itemize}
			\item SSID Name
			\item Wireless interface
		\end{itemize}
	\item Filter relevant packets based on SSID
	\item Capture probe request frames from nearby WiFi devices
	\item Allow device to connect to spoof access poin
	\item Send probe requests captured to the server
	\item Determine authentication type of previous network
	\item Bridge traffic from soft AP to real AP
\end{itemize}

\subsubsection*{Android Application}
The application must:-
\begin{itemize}
	\item Must emulate popular Android application
	\item Capture the users location data
	\item Capture a unique identifier for the device
	\item Send the captured data to the server with no encryption
\end{itemize}

\subsubsection*{TCP Server}
The application must:-
\begin{itemize}
	\item Accept a number of clients connecting and disconnecting
	\item Display the packets received
\end{itemize}

\subsubsection{Constraints}
The application will not be able to perform the honeypot attack on mobile devices that broadcast probe requests for encrypted networks. It will also be limited to the Operating System in which it has been developed to run on, in this instance that is Backtrack 5 Revision 3 or Kali. The application will only support one fake access point at a time due to the limitations imposed in order to meet ethical guidelines.

\subsubsection{Assumptions}
It is assumed that the Operating System running on the Virtual Machine will have access to the network that the host machine is connected to, and this will be connected via Ethernet.

\subsection{Specific Requirements}
This section justifies the requirements and lists them with a numerical value in order to track them through the design, implementation, and testing stages.

\label{requirements:specific}
\subsubsection{Honey Pot Application}
\begin{enumerate}
\item Take user defined parameters

The application shall accept an SSID to filter the the received probe request frames to ensure that data capturing is limited to devices owned by the user testing the application, but still allow testing of multiple devices at a time. It must also accept the wireless interface that the application is to act upon.

\item Filter packets based on SSID

As explained prior to this, the filtering enables the application to run in an ethically sound way.

\item Capture and parse probe request frames

The probe requests are the gateway to identifying vulnerable devices.

\item Allow device to connect to spoof access point

Once a device has been identified as vulnerable the application must spawn a fake access point using credentials found in the probe requests broadcast. This may either be implemented programmatically, or by take advantage of existing solutions.

\item Send probe requests captured to the server

The server may be capable of profiling passing device owners through SSIDs broadcast, so the honeypot application must support sending the MAC address and SSID of each probe request to the server using TCP.

\item Determine authentication type of previous network

To cut down on the number of wasted attempts at hijacking device traffic, the honeypot application must predetermine whether the device is searching for a network with open authentication.

\item  Bridge traffic from the mobile device

In order for the device to access the internet, and for the attacker to capture data packets, the fake access point must have their traffic bridged to an actual network with internet access.
\end{enumerate}
%%%%%%%%%%%%%%%%%%%%%%%%%%%%%%
\subsubsection{Android Application}
\begin{enumerate}
\item  Emulate a popular Android application

To demonstrate how easy it is for seemingly harmless applications to access and send personal data, the application must be similar to a currently popular application.

\item  Capture the users location data

As detailed in section \ref{intro:leaked-data} applications are increasingly sending information to advertisers in order to deliver personalised content. Location data is popular as it allows advertisers to deliver location-based advertisements, so, to mimic this, the application must report the device's current location. 

\item   Capture a unique identifier for the device

Future expansion may allow the server to profile unique devices. In order to facilitate this, the application must send a unique identifier with each message.

\item  Send the captured data to the server with no encryption

The application must send the data unencrypted as the whole reasoning behind this is to demonstrate an application that leaks data.
\end{enumerate}
%%%%%%%%%%%%%%%%%%%%%%%%%%%%%%
\subsubsection{TCP Server}
\begin{enumerate}
\item Accept a number of clients connecting and disconnecting

To support the application's implementation of a TCP client, and multiple devices, the server must be able to handle multiple connections and disconnections.

\item Display the packets received

The server must print each packet received in a user friendly manner.
\end{enumerate}
\clearpage