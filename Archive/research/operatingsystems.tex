\subsection{Operating Systems}
Windows is ruled out from the start as it has a number of severe limitations. The biggest limitation is the lack of support for injecting data packets. The project relies on injecting packets to create a soft access point, and will further limit any expansions that need to inject data packets to perform attacks such as HTTP Cache Poisoning.

With this in mind, there are Linux distributions that have been developed with the purpose of being used for network penetration testing that have various applications and tools pre-installed to take advantage of WiFi vulnerabilities. I also want to look at generic distributions with a mind to see whether the software will run on embedded platforms, with little porting, such as the IGEP V2 and the cheaper alternatives, the Beagleboard and Raspberry Pi.

For each distribution I tested firstly whether the Air-ng applications would install and run, as if they wouldn’t it was almost a guarantee that programs written using Lorcon would not work correctly. Once it was determined these would run, I tested a couple of wireless adaptors by setting up a soft AP, bridging this the actual hard AP, and forwarding traffic through it, using the Air-ng suite, as this would produce a similar effect to the final application.
\subsubsection{Backtrack 5 R3}
Backtrack was a security distribution that focused digital forensics and penetration testing. It comes pre-installed with a whole host of applications and tools, and was bootable from Live CD and USB. 

This distribution comes with the Air-ng applications pre-installed, which was a good guarantee that any attempts to use libraries based on libpcap will meet little resistance from install.

Results of the Air-ng test

Backtrack is unfortunately no longer supported; however, a branch- Kali Linux- has been developed and is actively updated.
\subsubsection{Kali}
Kali marked the switch from an Ubuntu base to Debian for the Backtrack developers. One of the big updates from Backtrack is further support for ARM devices, which means it can be run on the Raspberry Pi. This is a huge benefit as it would reduce any effort in porting the application to an OS running on the board.
\subsubsection{Ubuntu}

\subsubsection{Fedora}